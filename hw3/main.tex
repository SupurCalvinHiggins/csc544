\documentclass{article}
\usepackage{graphicx}

\usepackage{amsthm}
\usepackage{amsmath}
\usepackage{amssymb}
\usepackage{hyperref}

\renewcommand\qedsymbol{//}

\title{CSC 544 Homework 3}
\author{Calvin Higgins}
\date{September 2025}

\begin{document}

\maketitle

\section*{Problem 1}

\noindent
Consider the following model of cellphone conversations: \\

\noindent
We have an undirected graph $G = (V, E)$ where the vertices are people, and each edge indicates that two people are within range of each other. Whenever two 
people are talking, their neighbors must stay silent on that frequency to avoid interference. Thus, a set of conversations consists of a set of edges 
$C \subseteq E$, where the vertices in different edges in $C$ cannot be neighbors of each other. \\

\noindent
The cellphone capacity $c(G)$ is the largest number of conversations that can take place simultaneously on one frequency, i.e. the size of the largest such set
$C$. The \textbf{Cellphone Capacity Problem (CCP)} asks: given a graph $G$ and an integer $k$, is $c(G) \geq k$? \\

\noindent
Prove that \textbf{CCP} is in \textbf{NP-Complete}.

\subsection*{Solution}

\begin{proof}
    Clearly \textbf{CCP} is in \textbf{NP}. Given set of conversations $C$, we can check that no interference occurs and that $|C| \geq k$, so $c(G) \geq k$. 
    This takes polynomial time with respect to the number of vertices and edges. Now, we show that \textbf{CCP} is \textbf{NP-Hard}. \\

    \noindent
    We proceed by reduction from \textbf{3-Coloring}. Suppose $G = (V, E)$ is an instance of the \textbf{3-Coloring} problem. We aim to construct an instance of
    \textbf{CCP}, that is a graph $G' = (V', E')$ and integer $k$, such that $c(G') \geq k$ if and only if $G$ is 3-colorable. \\

    \noindent
    For each vertex $v \in V$, create a new $K_3$, with vertices $v_r, v_g$ and $v_b$. The purpose of this gadget is to encode a vertex and its color. Exactly
    one of the edges in each $K_3$ will be selected as part of the maximum conversation set. The color of $v$ is encoded by the vertex that is not incident to
    the selected edge. For example, if $\{v_g, v_b\}$ is selected as the edge, then the color of vertex $v$ is red. \\

    \noindent
    For each edge $\{u, v\} \in E$, create a new $K_{12}$ and twelve new edges uniquely pairing up each new vertex to a vertex in the $u$, $v$ vertex gadgets. 
    The purpose of this gadget is to encode edges and the color constraint. If the $u, v$ vertex gadgets encode the same coloring, none of the new edges can be 
    selected as part of the maximum conversation set. There are two cases. If an edge in the $K_{12}$ is taken, then no other edges incident to a vertex in the 
    $K_{12}$ can be taken. Then, at most two additional edges can be taken, one from each $K_3$. If an edge from a $K_3$ to the $K_{12}$ is taken, then no 
    other edges in that $K_3$ or the $K_{12}$ can be taken, so at most one additional edge in the other $K_3$ can be taken. In this manner, the $K_{12}$ ensures 
    that if valid coloring exists for $u, v$, they will be colored as such in the maximum conversation set, since otherwise only two edges will be added instead
    of three. \\

    \noindent
    Let $k = |E| + |V|$. The purpose of this gadget is to ensure that every vertex receives a color. Each vertex gadget can contribute at most one edge, and 
    each edge gadget can contribute at most one edge to the maximum conversation set. \\

    \noindent
    Clearly, this reduction is polynomial in $|V|$ and $|E|$. We aim to show that the reduction yields a yes instance if and only if $G$ is 3-colorable. \\

    \noindent
    First, we show that if $G$ is 3-colorable, then the reduction yields a yes instance. Suppose $G$ is 3-colorable. For each vertex, take the edge in the 
    gadget corresponding to its color. For example, if the vertex $v$ is red, take $\{v_g, v_b\}$. Then, consider each edge gadget. Since the incident vertices
    are different colors, there are four vertices, the color vertices from the $K_3$'s, and their neighbors in the $K_{12}$, such that the edge between the two 
    vertices in the $K_{12}$ can be taken, because neither of the color vertices are taken, and none of the other vertices in the $K_{12}$ are taken. Thus, the 
    total number of taken edges is $|V| + |E|$ as desired. \\

    \noindent
    Next, we show that if the reduction yields a yes instance, then $G$ is 3-colorable. Suppose the reduction yields a yes instance. Color each vertex in $G$ 
    according to the color of the corresponding vertex gadget. Since the reduction yields a yes instance, every edge gadget must have a taken edge, so the 
    colors of the vertex gadgets must be different. Hence this is valid 3-coloring. \\

    \noindent
    Since there is a polynomial time reduction from \textbf{3-Coloring} to \textbf{CCP}, and \textbf{3-Coloring} is in \textbf{NP-Hard}, 
    \textbf{CCP} is in \textbf{NP-Hard}. \\

    \noindent
    Since \textbf{CCP} is in \textbf{NP} and \textbf{NP-Hard}, it is in \textbf{NP-Complete}.
\end{proof}

\section*{Problem 2(a)}

\noindent
TLC Coffee has decided to expand to maximize profits over all of South County, by building new TLCs on many street corners. The management turns to URI’s 
Computer Science majors for help. The students come up with a representation of the problem: \\

\noindent
The street network is described as an arbitrary undirected graph $G = (V, E)$, where the potential restaurant sites are the vertices of the graph. Each vertex 
$u$ has a nonnegative integer value $p(u)$, which describes the potential profit of site $u$. Two restaurants cannot be built on adjacent vertices (to avoid 
self-competition). The students further come up with an exponential-time algorithm that chooses a set $U \subseteq V$, but it runs in $O(2^{|V|} |E|)$ time. The TLC 
ownership is dismayed, and is convinced they should have hired UConn’s CS majors instead. In order to prove that nobody else could have come up with a better 
algorithm, you need to prove the hardness of the \textbf{COFFEE} problem. \\

\noindent
Define \textbf{COFFEE} to be the following decision problem: given an undirected graph $G = (V, E)$, given a mapping $p$ from vertices $u \in V$ to nonnegative 
integer profits $p(u)$, and given a nonnegative integer $k$, decide whether there is a subset $U \subseteq V$ such that no two vertices in $U$ are neighbors in 
$G$, and such that $\sum\limits_{u \in U} p(u) \geq k$. \\

\noindent
Prove that \textbf{COFFEE} is in \textbf{NP-Hard}. 

\subsection*{Solution}

\begin{proof}
    \noindent
    We proceed by reduction from \textbf{3-Coloring}. Suppose $G = (V, E)$ is an instance of the \textbf{3-Coloring} problem. We aim to construct an instance of
    \textbf{COFFEE}, that is a graph $G' = (V', E')$, a profit function $p$ and integer $k$, such that it is a yes instance if and only if $G$ is 3-colorable. \\

    \noindent
    For each vertex $v \in V$, create a new $K_3$, with vertices $v_r, v_g$ and $v_b$. The purpose of this gadget is to encode a vertex and its color. A 
    coffee shop can be built on at most one of the vertices in each $K_3$. The color of $v$ is encoded by the chosen vertex. For example, if a coffee shop is 
    built on $v_r$, then the color of vertex $v$ is red. \\

    \noindent
    For each edge $\{u, v\} \in E$, create three new edges $\{u_r, v_r\}$, $\{u_g, v_g\}$ and $\{u_b, v_b\}$. The purpose of this gadget is to encode edges and
    the color constraint. The $u$ and $v$ gadgets cannot encode the same coloring, since otherwise, two coffee shops would be adjacent. \\

    \noindent
    Finally, take $k = |V|$. The purpose of this gadget is to ensure that every vertex is colored. Since there is at most one coffee shop per vertex gadget and
    there are $|V|$ vertex gadgets, this ensures every vertex gadget has at least one coffee shop, that is, the vertex is colored. \\

    \noindent
    More formally, let $V' = \bigcup\limits_{v \in V} \{ v_r, v_g, v_b \}$, 
    $E' = \bigcup\limits_{\{u, v\} \in E} \{ \{u_r, v_r\}, \{u_g, v_g\}, \{u_b, v_b\} \}$, $G' = (V', E')$, and $k = |V|$. Clearly, this reduction is
    polynomial in $|V|$ and $|E|$. We aim to show that $(G', k)$ is a yes instance if and only if $G$ is 3-colorable. \\

    \noindent
    First, we show that if $G$ is 3-colorable, then $(G', k)$ is a yes instance. Suppose $G$ is 3-colorable. 
    Take $U = \{ v_c \mid c \text{ is the color of } v \}$. By construction, $U$ contains exactly one vertex from each $K_3$ vertex gadget. Moreover, no two 
    vertices $u_c, v_c \in U$ are adjacent, since otherwise, the vertices $u, v \in V$ would have the same color $c$ but, by construction, $\{u, v\} \in E$, 
    which is impossible. Also $|U| = |V| \geq k = |V|$, so $(G', k)$ is a yes instance. \\

    \noindent
    Next, we show that if $(G', k)$ is a yes instance, then $G$ is 3-colorable. Suppose $(G', k)$ is a yes instance. Then there exists some set 
    $U \subseteq V'$, such that $U$ is valid and $|U| \geq k = |V|$. By construction, each of the $|V|$ vertex gadgets can have at most one coffee shop, so $U$
    must contain exactly $|V|$ coffee shops, one from each vertex gadget. Take the color of $v$ to be the ``color'' of the vertex selected in $v$'s gadget. 
    Then, we have a color for each vertex. For all edges $\{u, v\}$ in $G$, $G'$ contains edges between the same color vertices in $v$ and
    $u$'s gadgets, $u$ and $v$ must have different colors. Therefore, we have a valid 3-coloring of $G$. \\

    \noindent
    Since there is a polynomial time reduction from \textbf{3-Coloring} to \textbf{COFFEE}, and \textbf{3-Coloring} is in \textbf{NP-Hard}, \textbf{COFFEE} is 
    in \textbf{NP-Hard}.
\end{proof}

\section*{Problem 2(b)}

Explain why \textbf{COFFEE} in \textbf{NP-Hard} implies that, if there is a polynomial-time algorithm to solve \textbf{COFFEE}, i.e., to output a subset 
$U$ that maximizes the total profit, then \textbf{P} = \textbf{NP}.

\subsection*{Solution}

\begin{proof}
    Suppose there exists a polynomial-time algorithm to solve \textbf{COFFEE}. Fix some problem $\Pi \in \mathbf{NP}$. We will show that there is a 
    polynomial-time algorithm to solve $\Pi$. Since \textbf{COFFEE} is in \textbf{NP-Hard}, there is polynomial-time reduction from $\Pi$ to \textbf{COFFEE}. 
    Then, we can solve \textbf{COFFEE} in polynomial-time by the assumption. Since $\Pi$ can be solved in polynomial-time, $\mathbf{NP} \subseteq \mathbf{P}$, 
    so $\mathbf{P} = \mathbf{NP}$.
\end{proof}

\section*{Problem 3}

\noindent
\textbf{1-in-3-SAT} is a variant of \textbf{3-SAT} where each clause is satisfied if exactly one of its literals is true. So for instance, the sentence 
$(x,y, \neg z) \land (\neg x, \neg y, \neg z)$ could be satisfied by setting $x$ to true, $y$ to false, and $z$ to true. However, 
$(x,y,z) \land (\neg x, \neg y, \neg z)$ cannot be satisfied. \\

\noindent
Prove that \textbf{1-in-3-SAT} is in \textbf{NP-Complete}.

\subsection*{Solution}

\begin{proof}
    \noindent
    Clearly \textbf{1-in-3-SAT} is in \textbf{NP}. Given a satisfying assignment, we can check that the sentence is satisfied and that exactly one true literal 
    in each clause in true in polynomial-time with respect to the number of clauses. Now, we show that \textbf{1-in-3-SAT} is \textbf{NP-Hard}. \\

    \noindent
    We proceed by reduction from \textbf{3-Coloring}. Suppose $G = (V, E)$ is an instance of the \textbf{3-Coloring} problem. We aim to construct an instance of
    \textbf{1-in-3-SAT}, that is a set of literals $L$ and clauses $C$, such that it is a yes instance if and only if $G$ is 3-colorable. \\

    \noindent
    For each vertex $v \in V$, create literals $r_v, g_v$ and $b_v$ and clause $(r_v, g_v, b_v)$. The purpose of this gadget is to encode a vertex and its 
    color. Exactly one of $r_v, g_v$ and $b_v$ must be true. The color of $v$ is encoded by the true literal. For example, if $r_v$ is true, the color of 
    vertex $v$ is red. \\

    \noindent
    For each edge $\{u, v\} \in E$, create three new clauses $(r_u, r_v, x_{ruv})$, $(g_u, g_v, x_{guv})$ and  $(b_u, b_v, x_{buv})$. The purpose of this gadget
    is to encode edges and the color constraint. The $u$ and $v$ gadgets cannot encode the same coloring, since otherwise, two literals in a clause would be 
    true. The $x$ literals ensure that when neither vertex is a given color, the clause is still satisfiable. \\

    \noindent
    More formally, let 
    \begin{align*}
        L = \left( \bigcup\limits_{v \in V} \{ r_v, g_v, b_v \} \right) \cup \left( \bigcup\limits_{\{u, v\} \in E} \{x_{ruv}, x_{guv}, x_{buv}\} \right)
    \end{align*}
    and
    \begin{align*}
        C = \left( \bigcup\limits_{v \in V} \{ (r_v, g_v, b_v) \} \right) \cup \left( \bigcup\limits_{\{u, v\} \in E} \{(r_u, r_v, x_{ruv}), (g_u, g_v, x_{guv}), (b_u, b_v, x_{buv})\} \right)
    \end{align*}
    Clearly, this reduction is polynomial in $|V|$ and $|E|$. We aim to show that $(L, C)$ is a yes instance if and only if $G$ is 3-colorable. \\

    \noindent
    First, we show that if $G$ is 3-colorable, then $(L, C)$ is a yes instance. Suppose $G$ is 3-colorable. For each vertex $v$, set the literal corresponding 
    to its color to true, and the others to false. Then, all clauses without an $x$ must be true since each node has exactly one color. Consider each clause 
    with an $x$. Each clause either has exactly one true literal or zero since adjacent vertices have different colors. If the clause is false, set the $x$ to 
    true. Then, all clauses have exactly one true literal as desired. \\

    \noindent
    Next, we show that if $(L, C)$ is a yes instance, then $G$ is 3-colorable. Suppose $(L, C)$ is a yes instance. Then exactly one of the color literals 
    associated with vertex $v$ is true. Set $v$ to that corresponding color. Each node has a color, and adjacent nodes have different colors, since otherwise,
    one of the three edge clauses would have two true literals. \\

    \noindent
    Since there is a polynomial time reduction from \textbf{3-Coloring} to \textbf{1-in-3-SAT}, and \textbf{3-Coloring} is in \textbf{NP-Hard}, 
    \textbf{1-in-3-SAT} is in \textbf{NP-Hard}. \\

    \noindent
    Since \textbf{1-in-3-SAT} is in \textbf{NP} and \textbf{NP-Hard}, it is in \textbf{NP-Complete}.
\end{proof}

\section*{Problem 4}

Consider the following variant of \textbf{3-Coloring}: \\

\noindent
In addition to the graph $G$, we are given a forbidden color $f_v$ for each vertex $v$. For instance, vertex 1 may not be red, vertex 2 may not be green, and so
on (with still a total palette of 3 colors). We can ask whether $G$ has a proper 3-coloring in which no vertex is colored with its forbidden color. \\

\noindent
The \textbf{Constrained-3-Coloring} problem asks, given a graph $G = (V, E)$ with $n$ vertices and a list $f_1, \ldots, f_n \in \{1,2,3\}$, determine if $G$ has
a proper 3-coloring $c_1, \ldots, c_n$ where $c_v \neq f_v$ for all vertices $v \in V$. \\

\noindent
Show \textbf{Constrained-3-Coloring} is in \textbf{NP-Complete} or show that \textbf{Constrained-3-Coloring} is in \textbf{P}.

\subsection*{Solution}

\begin{proof}
    \noindent
    We will show that \textbf{Constrained-3-Coloring} is in \textbf{P}. \\

    \noindent
    We proceed by reduction to \textbf{2-SAT}. Suppose $G = (V, E)$ where $f_v$ is the restricted color for vertex $v$. We aim to construct an instance of 
    \textbf{2-SAT} such that $G$ is colorable if and only if the \textbf{2-SAT} instance is satisfiable. \\

    \noindent
    For each vertex $v \in V$, arbitrarily label its two possible colors $a$ and $b$. Create two variables $v_a, v_b$, and two clauses $(v_a \lor v_b)$ and 
    $(\neg v_a \lor \neg v_b)$. The purpose of this gadget is to represent vertex colors and their restriction. Exactly one of $v_a$ and $v_b$ must be true: 
    this is the encoded color of $v$. \\

    \noindent
    For each edge $\{u, v\} \in E$ and color $c$, create a clause $(\neg u_c \lor \neg v_c)$. The purpose of this gadget is to enforce the restriction that two 
    adjacent nodes cannot have the same color.  \\

    \noindent
    Clearly, this reduction is polynomial in $|V|$ and $|E|$. We aim to show that $G$ is colorable if and only if the \textbf{2-SAT} instance is satisfiable

    \noindent
    First, we show that if $G$ is colorable, then \textbf{2-SAT} instance is satisfiable. Suppose $G$ is colorable. For each vertex $v \in V$, set $v_a, v_b$ 
    according to the color of $v$. Then, the vertex gadget clauses are true because exactly one of $v_a, v_b$ will be true. Also, the edge clauses will be
    true since no adjacent vertices have the same color. Hence, the \textbf{2-SAT} instance is satisfiable as desired. \\

    \noindent
    Next, we show that if \textbf{2-SAT} instance is satisfiable, then $G$ is colorable. Suppose \textbf{2-SAT} instance is satisfiable. For each $v_a, v_b$,
    by the vertex clause gadget, exactly one of $v_a, v_b$ is true. Set $v$ to the corresponding coloring. Then every vertex in $G$ has a color, and no two 
    adjacent vertices have the same color, since otherwise, some clause $(\neg u_c \lor \neg v_c)$ would be false. Hence $G$ is colorable as desired. \\

    \noindent
    Since there is a polynomial time reduction from \textbf{Constrained-3-Coloring} to \textbf{2-SAT}, and \textbf{2-SAT} is in \textbf{P}, 
    \textbf{Constrained-3-Coloring} is in \textbf{P}.
\end{proof}

\end{document}