\documentclass{article}
\usepackage{graphicx}

\usepackage{amsthm}
\usepackage{amsmath}
\usepackage{amssymb}
\usepackage{hyperref}

\renewcommand\qedsymbol{//}

\title{CSC 544 Homework 2}
\author{Calvin Higgins}
\date{September 2025}

\begin{document}

\maketitle

\noindent
As discussed in class, I assume that function juxtaposition is right-associative. I referred to~\url{https://j-hui.com/pages/normal-forms/} for the definitions 
of beta normal form and weak normal form.

\section*{Problem 1}

Find a beta normal form for $(\lambda(x,y).(x,y,z))(4,3)$.

\subsection*{Solution}

We have that
\begin{align*}
    (\lambda(x,y).(x,y,z))(4,3) &= ((\lambda x. \lambda y.(x,y,z)) 4) 3 \\
    &= (\lambda y.(4,y,z)) 3 \\
    &= (4,3,z) \\
\end{align*}

\section*{Problem 2}

Find a beta normal form for $(\lambda(x,y).y)(5,3)$.

\subsection*{Solution}

We have that 
\begin{align*}
    (\lambda(x,y).y)(5,3) &= ((\lambda x .\lambda y.y)5)3 \\
    &= (\lambda y.y)3 \\
    &= 3
\end{align*}

\section*{Problem 3}

Find a beta normal form for $(\lambda x.x)(\lambda x.x)$

\subsection*{Solution}

We have that 
\begin{align*}
    (\lambda x.x)(\lambda x.x) &= \lambda x.x
\end{align*}

\section*{Problem 4}

Find a beta normal form for $(\lambda x.(xx))(\lambda x.(xx))$.

\subsection*{Solution}

There is no beta normal form. After the $k$-th $\beta$-reduction, it is $(\lambda x.(xx))(\lambda x.(xx))$. Likewise, there is no weak normal form.

\section*{Problem 5}

Let $::$ be the concatenation operator. Find a beta normal form for $(\lambda (x :: q).q)(f :: u :: n :: [])$.

\subsection*{Solution}

We have that 
\begin{align*}
    (\lambda (x :: q).q)(f :: u :: n :: []) &= u :: n :: []
\end{align*}

\section*{Problem 6}

Let $c\;?\;t : f$ be the ternary operator. Find a beta normal form for $(\lambda xyz.x > 0 \;?\; y : z)3(\lambda q.q-1)(\lambda p.p + 1) 1$.

\subsection*{Solution}

We have that
\begin{align*}
    (\lambda xyz.x > 0 \;?\; y : z)3(\lambda q.q-1)(\lambda p.p + 1) 1 
    &= ((\lambda x .\lambda y .\lambda z.x > 0 \;?\; y : z)3)((\lambda q.q-1)((\lambda p.p + 1)1)) \\
    &= ((\lambda x .\lambda y .\lambda z.x > 0 \;?\; y : z)3)((\lambda q.q-1)2) \\
    &= ((\lambda x .\lambda y .\lambda z.x > 0 \;?\; y : z)3)1 \\
    &= (\lambda y .\lambda z.3 > 0 \;?\; y : z)1 \\
    &= (\lambda y .\lambda z.y)1 \\
    &= (\lambda y.y)1 \\
    &= 1 \\
\end{align*}

\section*{Problem 7}

Find a beta normal form for $Ya$ where $Y = \lambda f.(\lambda x.f(xx))(\lambda x.f(xx))$.

\subsection*{Solution}

We have that
\begin{align*}
    Ya &= (\lambda f.(\lambda x.f(xx))(\lambda x.f(xx)))a \\
    &= (\lambda x.a(xx))(\lambda x.a(xx)) \\
    &= a((\lambda x.a(xx)) (\lambda x.a(xx))) \\
    &= a((\lambda f.(\lambda x.f(xx)) (\lambda x.f(xx)))a) \\
    &= aYa \\
\end{align*}

\noindent
There is no beta normal form. After the $k$-th $\beta$-reduction, it is $a^k Y a$. However, $aYa$ is a weak normal form.

\section*{Problem 8}

Find a beta normal form for $Y(\lambda f.(\lambda x.(x=1)\;?\;1 : \mathrm{mult}(x, f(x-1))))$ where $Y = \lambda f.(\lambda x.f(xx))(\lambda x.f(xx))$.

\subsection*{Solution}

Let $F = \lambda f.(\lambda x.(x=1)\;?\;1 : \mathrm{mult}(x, f(x-1)))$. By Problem 7, we have that
\begin{align*}
    YF &= FYF \\
    &= (\lambda f.(\lambda x.(x=1)\;?\;1 : \mathrm{mult}(x, f(x-1))))(YF) \\
    &= \lambda x.(x=1)\;?\;1 : \mathrm{mult}(x, YF(x-1))
\end{align*}

\noindent
Like Problem 7, there is no beta normal form. However, $\lambda x.(x=1)\;?\;1 : \mathrm{mult}(x, YF(x-1))$ is a weak normal form.

\end{document}