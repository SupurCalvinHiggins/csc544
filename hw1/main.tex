\documentclass{article}
\usepackage{graphicx}

\usepackage{amsthm}
\usepackage{amsmath}
\usepackage{amssymb}

\renewcommand\qedsymbol{//}

\title{CSC 544 Homework 1}
\author{Calvin Higgins}
\date{September 2025}

\begin{document}

\maketitle

\section*{Problem 1}

Let $Z = \{ 0 1^k 0 \mid k \geq 0 \}$.

\subsection*{Part (a)}

Give a GNFA that recognizes $Z$. \\

\noindent
The GNFA has two states: one is the start state, the other is the accept state. There is a single arrow from the start state to the accept state labeled with 
the regular expression $01^*0$. 

\subsection*{Part (b)}

Give a regular expression that generates $Z$. \\

\noindent
The expression $01^*0$ generates $Z$.

\section*{Problem 2}

\subsection*{Part (a)}

Let $L = \{ \text{abra}(\text{cad})^k\text{abra} \mid k \geq 1 \}$. Prove or disprove: $L$ is regular.

\begin{proof}
    We will show that $L$ is regular by construction. Let $A = \{ \text{abra} \}$ and $C = \{ \text{cad} \}$. Since $A$ and $C$ have finite cardinality, they 
    are regular. Define $L' = A \circ C \circ C^* \circ A$. Since $A$ and $C$ are regular, and the class of regular languages is closed under $\circ$ and $*$, 
    $L'$ is regular. \\
    
    \noindent
    We will argue that $L' = L$. Note that
    \begin{align*}
        L' &= A \circ C \circ C^* \circ A \\
        &= \{ \text{abra} \} \circ \{ \text{cad} \} \circ \{ (\text{cad})^k \mid k \geq 0 \} \circ \{ \text{abra} \} \\
        &= \{ \text{abra} \} \circ \{ (\text{cad})^k \mid k \geq 1 \} \circ \{ \text{abra} \} \\
        &=  \{ \text{abra} (\text{cad})^k \text{abra} \mid k \geq 1 \} \\
        &= L
    \end{align*} 
    so $L$ is regular as desired.
\end{proof}

\subsection*{Part (b)}

Let $L = \{ (\text{abra})^r (\text{cad})^k (\text{abra})^r \mid k \geq 1, r \geq 0 \}$. Prove or disprove: $L$ is regular.

\begin{proof}
    We will show that $L$ is not regular via the pumping lemma. Assume that $L$ is regular, and let $p$ be the pumping length. Take 
    \begin{align*}
        s = (\text{abra})^p (\text{cad}) (\text{abra})^p 
    \end{align*}
    so $s \in L$. \\

    \noindent
    We will show that $s$ cannot be pumped. For any choice of $u, v, w$ such that $uvw = s$, $|uv| \leq p$ and $|v| \geq 1$, $v$ contains some fragment of some 
    abra to the left of cad. Pumping down, we have that $s' = uv^0w = uw \in L$. However, $s'$ contains too few symbols to form enough abra's on the left
    to match the $p$ abras on the right (\textit{the casework is too painful for anything but hand waving}), so $s' \not \in L$, which is impossible. Hence $L$
    is not regular.
\end{proof}

\section*{Problem 3}

Let $\Sigma = \{ 0, 1 \}$.

\subsection*{Part (a)}

Let $L = \{ xyx \mid x, y \in \Sigma^* \}$. Prove that $L$ is regular.

\begin{proof}
    We will show that $L$ is regular by construction. The main idea is that $L$ contains all strings (e.g. take $x = \epsilon$). Since $\Sigma$ has finite 
    cardinality, $\Sigma$ is regular. Define $L' = \Sigma^*$. Since $\Sigma$ is regular, and the class of regular languages is closed under $*$, $L'$ is 
    regular. \\
    
    \noindent
    We will argue that $L' = L$. Clearly $L \subseteq L' = \Sigma^*$. Fix some $s \in L'$. Let $x = \epsilon$ and $y = s$. Then $x \in \Sigma^*$ and 
    $y \in L' = \Sigma^*$. By definition, $xyx \in L$, so $xyx = \epsilon s \epsilon = s \in L$. Therefore, $L' \subseteq L$. Hence, $L' = L$, so $L$ is regular 
    as desired.
\end{proof}

\subsection*{Part (b)}

Let $L = \{ xyxy \mid x, y \in \Sigma^* \}$. Prove that $L$ is not regular. 

\begin{proof}
    We will show that $L$ is not regular via the pumping lemma. Assume that $L$ is regular, and let $p$ be the pumping length. Take $x = 0^p 1^p$ and 
    $y = 0^p 1^p$. Then $x, y \in \Sigma^*$, so $s = xyxy \in L$. \\

    \noindent
    We will show that $s$ cannot be pumped. For any choice of $u, v, w$ such that $uvw = s$, $|uv| \leq p$ and $|v| \geq 1$, $v = 0^k$ for some $1 \leq k \leq p$ since
    the first $p$ symbols in $s$ are all zero. By the pumping lemma, $s' = uv^0w = 0^{p - k} 1^p 0^p 1^p \in L$. Suppose there exists some choice of $x', y'$ 
    such that $s' = x' y' x' y'$. Then, the first half of $s'$ is $x' y' = 0^{p - k}1^{p} 1^{k/2}$ and the second half of $s'$ is $x'y' = 1^{p-k/2}0^p$. 
    However, the first half is not the same as the second half, so $x' y' \neq x' y'$ which is impossible. Thus, there is no choice of $x' y'$ such that  
    $s' = x' y' x' y'$, so $s' \not \in L$, which is impossible. Hence $L$ is not regular as desired.
\end{proof}

\section*{Problem 4}

\subsection*{Part (a)}

Prove or disprove: the class of non-regular languages is closed under union.

\begin{proof}
    We will show that the class of non-regular languages is not closed under union by counterexample. The main idea is that the complement of a non-regular 
    language is non-regular, but the union of any language and its complement is regular. \\

    \noindent
    First, we will show that class of non-regular languages is closed under complement. For the sake of contradiction, suppose not. Then, there exists some 
    non-regular language $L$ such that $L^c$ is regular. However, since the class of regular languages is closed under complement, $(L^c)^c = L$ is regular, 
    which is impossible. Therefore, the class of non-regular languages is closed under complement. \\

    \noindent
    Now, we will show that the class of non-regular languages is not closed under union. Fix some non-regular language $L$. Then $L^c$ is also non-regular.
    However, $L \cup L^c = \Sigma^*$ which is regular.
\end{proof}

\subsection*{Part (b)}

Prove or disprove: the class of non-regular languages is closed under concatenation.

\begin{proof}
    We will show that the class of non-regular languages is not closed under concatenation by counterexample. The main idea is that the set of unary strings 
    with perfect square lengths is non-regular, but any integer can be written as the sum of four perfect squares, so any unary string can be written as the 
    concatenation of four strings of perfect square length. \\

    \noindent
    Let $L = \{1^{n^2} \mid n \geq 0\}$ where $\Sigma = \{ 1 \}$. We will show that $L$ is non-regular via the pumping lemma. Assume that $L$ is regular, and
    let $p$ be the pumping length. Take $s = 1^{(p+1)^2}$. \\
    
    \noindent
    We will show that $s$ cannot be pumped. For any choice of $u,v,w$ such that $uvw = s$, $|uv| \leq p$ and $|v| \geq 1$, $v = 1^k$ for some $1 \leq k \leq p$.
    By the pumping lemma, $s' = uv^0w = 1^{(p+1)^2 - k} \in L$. Since $k \geq 1$, $(p+1)^2 - k \leq (p+1)^2 - 1 < (p+1)^2$. Since $k \leq p$, 
    $(p+1)^2 - k \geq (p^2 + 2p + 1) - p > p^2$. Since $p^2 < (p+1)^2 - k < (p+1)^2$, $(p+1)^2 - k$ is not a perfect square, so $s' \not \in L$. Hence $s$ 
    cannot be pumped, so $L$ is non-regular. \\

    \noindent
    Now, we will show that the class of non-regular languages is not closed under concatenation. By Lagrange's four square theorem, 
    \begin{align*}
        L \circ L \circ L \circ L &= \{1^{a^2} \mid a \geq 0 \} \circ\{1^{b^2} \mid b \geq 0 \} \circ \{1^{c^2} \mid c \geq 0 \} \circ \{1^{d^2} \mid d \geq 0 \} \\
        &= \{1^{a^2} 1^{b^2} 1^{c^2} 1^{d^2} \mid a, b, c, d \geq 0 \} \\
        &= \{1^{a^2 + b^2 + c^2 + d^2} \mid a, b, c, d \geq 0 \} \\
        &= \{1^n \mid n \geq 0\} \\
        &= \Sigma^*
    \end{align*}
    which is regular. 
\end{proof}

\subsection*{Part (c)}

Prove or disprove: the class of non-regular languages is closed under star.

\begin{proof}
    We will show that the class of non-regular languages is not closed under star. The main idea is that non-regular languages remain non-regular under union 
    with $\Sigma$, but then star generates $\Sigma^*$. \\

    \noindent
    Let $L$ be some non-regular language with alphabet $\Sigma$, and let $\Sigma^{-} = \Sigma - L$. Take $L' = L \cup \Sigma$. We will show that $L'$ is 
    non-regular by contradiction. For the sake of contradiction, suppose that $L'$ is regular. Since $\Sigma^{-}$ is finite, $\Sigma^{-}$ is regular. Therefore,
    $(\Sigma^{-})^c$ is regular. Since the class of regular languages is closed under intersection, $L' \cap (\Sigma^{-})^c$ is regular. However,
    \begin{align*}
        L' \cap (\Sigma^{-})^c &= (L \cup \Sigma^{-}) \cap (\Sigma^{-})^c \\
        &= (L \cap (\Sigma^{-})^c) \cup (\Sigma^{-} \cap (\Sigma^{-})^c) \\
        &= (L \cap (\Sigma - L)^c) \cup \emptyset \\
        &= L
    \end{align*}
    which is impossible. Therefore, $L'$ is non-regular. \\
    
    \noindent
    Now, we will show that the class of non-regular languages is not closed under star. We have that 
    \begin{align*}
        (L')^* = (L \cup \Sigma)^* = (L^* \cup \Sigma^*)^* = (\Sigma^*)^* = \Sigma^*
    \end{align*}
    which is regular.
\end{proof}

\section*{[Misread] Problem 4}

\subsection*{Part (a)}

Prove or disprove: the class of regular languages is closed under union.

\begin{proof}
    Let $L_1$ and $L_2$ be regular languages. We will show that $L = L_1 \cup L_2$ is regular by construction. The main idea is to simulate $M_1$ and $M_2$ 
    simultaneously, and accept if either accepts. \\
    
    \noindent
    Since $L_1$ is regular, there exists some DFA $M_1 = (Q_1, \Sigma_1, \delta_1, q_{0}^1, F_1)$ that recognizes $L_1$, and since $L_2$ is regular, there 
    exists some DFA $M_2 = (Q_2, \Sigma_2, \delta_2, q_{0}^2, F_2)$ that recognizes $L_2$. Define DFA 
    \begin{align*}
        M = ((Q_1 \cup q_{\text{reject}}) \times (Q_2 \cup q_{\text{reject}}), \Sigma_1 \cup \Sigma_2, \delta, (q_{0}^1, q_{0}^2), (F_1 \times Q_2) \cup (Q_1 \times F_2))
    \end{align*}
    where
    \begin{align*}
        \delta(s_k, (q_i^1, q_j^2)) = \begin{cases}
            (\delta_1(q_i^1), \delta_2(q_j^2)) & \text{ if } s_k \in \Sigma_1 \land s_k \in \Sigma_2 \land q_i^1 \in Q_1 \land q_j^2 \in Q_2 \\
            (\delta_1(q_i^1), q_{\text{reject}}) & \text{ if } s_k \in \Sigma_1 \land s_k \not \in \Sigma_2 \land q_i^1 \in Q_1 \\
            (q_{\text{reject}}, \delta_2(q_j^2)) & \text{ if } s_k \not \in \Sigma_1 \land s_k \in \Sigma_2 \land q_j^2 \in Q_2 \\
            (q_{\text{reject}}, q_{\text{reject}}) & \text{ otherwise}
        \end{cases}
    \end{align*}

    \noindent
    We will show that $M$ recognizes $L$. First, we show that if $s \in L$, then $M$ accepts $s$. Then, we show that if $M$ accepts $s$, then $s \in L$. \\ 
    
    \noindent
    Fix some $s \in L$ such that $s = s_1 s_2 \ldots s_m$. Define $r_0 = (q_0^1, q_0^2)$, and $r_{i+1} = \delta(s_i, r_i)$ for $0 \leq i \leq m - 1$. Since 
    $s \in L$, either $s \in L_1$ or $s \in L_2$. Assume $s \in L_1$. Then $M_1$ accepts $s$, so there exists some $r_0^1 = q_0^1$, 
    $r_{i+1}^1 = \delta_1(s_i, r_i^1)$ for $0 \leq i \leq m - 1$, where $r_m^1 \in F_1$. Consider the sequence of states $r_0, r_1, \ldots, r_m$ taken by $M$ on 
    $s$. Since the first component of $r_0$ is $q_0^1$, by definition of $M$, the first component of each $r_i$ is $r_i^1$. Thus, the first component of $r_m$ 
    is $r_m^1$, so $r_m \in F_1 \times Q_2$. By definition, $r_m$ is an accept state, so $M$ accepts $s$. The case where $s \in L_2$ is similar. \\

    \noindent
    Fix some $s$ accepted by $M$ such that $s = s_1 s_2 \ldots s_m$. Then, there exists some $r_0 = (q_0^1, q_0^2)$ and $r_{i+1} = \delta(s_i, r_i)$ for 
    $0 \leq i \leq m - 1$ such that $r_m = (q_j^1, q_k^2)$ where $q_j^1 \in Q_1$ or $q_k^2 \in Q_2$. Assume $q_j^1 \in Q_1$. By definition of $M$, there exists 
    $r_0^1 = q_0^1$, $r_{i+1}^1 = \delta_1(s_i, r_i^1)$ for $0 \leq i \leq m - 1$, such that $r_{m}^1 = q_j^1 \in Q_1$, namely $r_i^1$ is the first component of
    each $r_i$. Thus, $M_1$ accepts $s$, so $s \in L_1 \subseteq L$. The case where $q_k^2 \in Q_2$ is similar. \\

    \noindent
    Thus $M$ accepts $s$ if and only if $s \in L$. Hence $L$ is regular as desired.
\end{proof}

\subsection*{Part (b)}

Prove or disprove: the class of regular languages is closed under concatenation.

% idea: nfa, guess the split point starting from accept states in M_1
% eps arrow from M_1 accept to M_2 start
\begin{proof}
    Let $L_1$ and $L_2$ be regular languages. We will show that $L = L_1 \circ L_2$ is regular by construction. The main idea is to non-deterministically guess 
    a split $s = a b$ such that $a \in L_1$ and $b \in L_2$. We start in the machine recognizing $L_1$, $\epsilon$-move from the accept states in the machine 
    recognizing $L_1$ to the start state in the machine recognizing $L_2$, and accept if the machine recognizing $L_2$ accepts.  \\

    \noindent
    Since $L_1$ is regular, there exists some NFA $M_1 = (Q_1, \Sigma_1, \delta_1, q_0^1, F_1)$ that recognizes $L_1$, and since $L_2$ is regular, there exists
    some NFA $M_2 = (Q_2, \Sigma_2, \delta_2, q_0^2, F_2)$ that recognizes $L_2$. Define NFA
    \begin{align*}
        M = (Q_1 \cup Q_2, \Sigma_1 \cup \Sigma_2, \delta, q_0^1, F_2)
    \end{align*}
    where
    \begin{align*}
        \delta'_1(s_k, q_i) &= \begin{cases}
            \delta_1(s_k, q_i) & \text{ if } q_i \in Q_1 \land s_k \in \Sigma_1 \cup \{ \epsilon \} \\
            \emptyset & \text{ otherwise}
        \end{cases} \\
        \delta'_2(s_k, q_i) &= \begin{cases}
            \delta_2(s_k, q_i) & \text{ if } q_i \in Q_2 \land s_k \in \Sigma_2 \cup \{ \epsilon \} \\
            \emptyset & \text{ otherwise}
        \end{cases} \\
        \delta(s_k, q_i) &= \begin{cases}
            \delta'_1(s_k, q_i) \cup \{ q_0^2 \} & \text{ if } q_i \in F_1 \land s_k = \epsilon \\
            \delta'_1(s_k, q_i) \cup \delta'_2(s_k, q_i) & \text{ otherwise}
        \end{cases}
    \end{align*}
    We will show that $M$ recognizes $L$. First, we show that if $s \in L$, then $M$ accepts $s$. Then, we show that if $M$ accepts $s$, then $s \in L$. \\

    \noindent
    Fix some $s \in L$ where $s = s_1 s_2 \ldots s_m$. Then, there exists $a \in L_1$ and $b \in L_2$ such that $s = a b$. Since $a \in L_1$, $M_1$ 
    accepts $a$. Therefore, there exists some $r_0^1 = q_0^1$, $r_i^1 \in \delta_1(a_i, r_{i-1}^1)$ for $1 \leq i \leq |a|$, where $r_{|a|}^1 \in F_1$. Likewise, 
    $M_2$ accepts $b$. Therefore, there exists some $r_0^2 = q_0^2$, $r_i^2 \in \delta_2(b_i, r_{i-1}^2)$ for $1 \leq i \leq |b|$, where $r_{|b|}^2 \in F_2$. 
    Define $s' = a \epsilon b = s$, and $r = r_1 r_2 \ldots r_{|s'|}$ where
    \begin{align*}
        r_{i} = \begin{cases}
            r_0^1 & \text{ if } i = 0 \\
            r_i^1 & \text{ if } 1 \leq i \leq |a| \\
            r_0^2 & \text{ if } i = |a| + 1 \\
            r_i^2 & \text{ if } |a| + 2 \leq i \leq |s'|
        \end{cases}
    \end{align*}
    Now, we must verify that $r$ is an accepting computation of $M$ on $s'$:
    \begin{enumerate}
        \item We have that $r_0 = r_0^1 = q_0^1$ which is the start state of $M$.
        \item We consider three subcases
        \begin{enumerate}
            \item Assume $1 \leq i \leq |a|$. Then 
            \begin{align*}
                r_i = r_i^1 \in \delta_1(a_i, r_{i-1}^1) = \delta'_1(s'_i, r_{i-1}) \subseteq \delta(s'_i, r_{i-1})
            \end{align*}

            \item Assume $i = |a| + 1$. Then 
            \begin{align*}
                r_i = r_0^2 = q_0^2 \in \delta(\epsilon, r_{|a|}^1) = \delta(s'_{|a|}, r_{i-1}) 
            \end{align*}
            since $r_{|a|}^1 \in F_1$.

            \item Assume $|a| + 2 \leq i \leq |s'|$. Then
            \begin{align*}
                r_i = r_i^2 \in \delta_2(b_i, r_{i-|a|-2}^2) = \delta'_1(s'_i, r_{i-1}) \subseteq \delta(s'_i, r_{i-1})
            \end{align*}
        \end{enumerate}
        so for all $1 \leq i \leq |s'|$, $r_i \in \delta(s'_i, r_{i-1})$.
        \item We have that $r_{|s'|} = r_{|b|}^2 \in F_2$, so the final state is an accept state.
    \end{enumerate}
    Therefore, $M$ accepts $s$. \\

    \noindent
    Fix some $s = s_1 s_2 \ldots s_m$ where $M$ accepts $s$. Then, there exists some accepting sequence of states $r = r_0 r_1 \ldots r_m$. Since 
    $r_m \in F_2 \subseteq Q_2$, there exists $r_i \in Q_2$. Let $r_{t+1}$ be the first such $r_i \in Q_2$. Then $r_t \in Q_1$ so $s_{t+1} = \epsilon$, since 
    the states in $Q_2$ are only reachable from the states in $Q_1$ by $\epsilon$-moves. Define $a = s_1 \ldots s_{t}$ and $b = s_{t+2} \ldots s_{m}$. Then
    $s = a \epsilon b = a b$. We must show that $a \in L_1$ and $b \in L_1$. First, we show that $a \in L_1$. Define 
    \begin{align*}
        r^1 = r_0^1 r_1^1 \ldots r_{t}^1 = r_0 r_1 \ldots r_{t}
    \end{align*}
    We must show that $r^1$ is an accepting sequence of states for $M_1$:
    \begin{enumerate}
        \item We have that $r^1_0 = r_0 = q_0^1$ which is the start state.
        \item We have that $r^1_i = r_i \in \delta(s_i, r_{i-1})$, and by choice of $t$, $r_{i-1} \in Q_1$, so 
        \begin{align*}
            \delta(s_i, r_{i-1}) = \delta'_1(s_i, r_{i-1}) = \delta_1(s_i, r_{i-1}) = \delta_1(s_i, r_{i-1}^1)
        \end{align*}
        Thus, $r^1_i \in \delta_1(s_i, r_{i-1}^1)$.
        \item We have that $r_{t}^1 = r_{t} \in F_1$. Because transitions from states in $Q_1$ to $Q_2$ can only occur from 
        states in $F_1$, if $r_t \not \in F_1$, $r_{t+1} \not \in Q_2$ which is impossible. Thus, $r_t$ is an accept state.  
    \end{enumerate}
    The proof that $b \in L_2$ proceeds similarly. Therefore, $s = a\epsilon b = ab \in L$. \\

    \noindent
    Thus $M$ accepts $s$ if and only if $s \in L$. Hence $L$ is regular as desired.
\end{proof}

\subsection*{Part (c)}

Prove or disprove: the class of regular languages is closed under star.

\begin{proof}
    Let $L_1$ be a regular language. We will show that $L = L_1^*$ by construction. The main idea is to non-deterministically split $s = a^1 a^2 \ldots a^m$ 
    where $a^1, a^2, \ldots, a^m \in L_2$ ($a^{i}$ is the $i$-th word, not $a$ repeated $i$ times). We define a new start state as the only accept state, and
    $\epsilon$-move from the new start state to the old start state, and from each old accept state to the new start state. \\

    \noindent
    Since $L_1$ is regular, there exists some NFA $M_1 = (Q_1, \Sigma_1, \delta_1, q_0^1, F_1)$ that recognizes $L_1$. Define NFA
    \begin{align*}
        M = (Q_1 \cup \{ q_{\text{start}} \}, \Sigma_1, \delta, q_{\text{start}}, \{ q_{\text{start}} \})
    \end{align*}
    where
    \begin{align*}
        \delta'_1(s_k, q_i) &= \begin{cases}
            \delta_1(s_k, q_i) & \text{ if } q_i \in Q_1 \land s_k \in \Sigma_1 \cup \{ \epsilon \} \\
            \emptyset & \text{ otherwise}
        \end{cases} \\
        \delta(s_k, q_i) &= \begin{cases}
            \{ q_0^1 \} & \text{ if } q_i = q_{\text{start}} \land s_k = \epsilon \\
            \delta'_1(s_k, q_i) \cup \{ q_{\text{start}} \} & \text{ if } q_i \in F_1 \land s_k = \epsilon \\
            \delta'_1(s_k, q_i) & \text{ otherwise}
        \end{cases}
    \end{align*}
    We will show that $M$ recognizes $L$. First, we show that if $s \in L$, then $M$ accepts $s$. Then, we show that if $M$ accepts $s$, then $s \in L$. Since 
    I'm out of time, and have (hopefully) sufficiently demonstrated that I can grind out DFA proofs, I will just give a proof sketch for both directions. \\

    \noindent
    Fix some $s \in L$. Since $s \in L$, $s = a^1 a^2 \ldots a^m$ where each $a^i \in L_1$. Since $a^i \in L_1$, there is an accepting sequence of states $r^i$ 
    in $M_1$. Define $r = r^1 r^2 \ldots r^m$. Then, take $s' = \epsilon a^1 \epsilon \epsilon a^2 \epsilon \epsilon \ldots \epsilon \epsilon a^m \epsilon = s$. It can be shown that $r$ is an accepting
    sequence of states for $M$ on $s'$, so $M$ accepts $s' = s$. \\

    \noindent
    Fix some $s$ where $M$ accepts $s$. Since $M$ accepts $s$, there is an accepting sequence of states $r = r_0 r_1 \ldots r_{m}$. Split $r$ into the maximum 
    number of segments $r^1, r^2, \ldots r^n$, such that 
    $r = q_{\text{start}} r^1 q_{\text{start}} r^2 q_{\text{start}} \ldots q_{\text{start}} r^n q_{\text{start}}$. Then, it can be shown that each 
    $r^1, r^2 \ldots r^n$ is an accepting sequence for $M_1$ given the corresponding substring in $a^1, a^2, \ldots a^n$, where 
    $s' = \epsilon a^1 \epsilon \epsilon a^2 \epsilon \epsilon \ldots \epsilon \epsilon a^n \epsilon = s$. Thus $a^1, a^2, \ldots, a^n \in L_1$, so $s \in L$. \\

    \noindent
    Thus $M$ accepts $s$ if and only if $s \in L$. Hence $L$ is regular as desired.
\end{proof}

\end{document}